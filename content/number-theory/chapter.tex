\chapter{Number theory}

\section{Modular arithmetic}
	Binary representation of 30941 modulo $2^{2^k}$ ($k \in \{3,4,5,6,7\}$), treated as polynomial is irreducible.
	%\kactlimport{ModularArithmetic.h}
	\kactlimport{ModInverse.h}
	\kactlimport{ModPow.h}
	\kactlimport{ModLog.h}
	\kactlimport{ModSum.h}
	\kactlimport{ModMulLL.h}
	\kactlimport{ModSqrt.h}
	% \kactlimport{GF2k.h}

\section{Primality}
	\kactlimport{FastEratosthenes.h}
	\kactlimport{MillerRabin.h}
	\kactlimport{Factor.h}

\section{Divisibility}
	\kactlimport{euclid.h}
	% \kactlimport{Euclid.java}
	\kactlimport{CRT.h}

	% \subsection{Bézout's identity}
	% For $a \neq $, $b \neq 0$, then $d=gcd(a,b)$ is the smallest positive integer for which there are integer solutions to
	% $$ax+by=d$$
	% If $(x,y)$ is one solution, then all solutions are given by
	% $$\left(x+\frac{kb}{\gcd(a,b)}, y-\frac{ka}{\gcd(a,b)}\right), \quad k\in\mathbb{Z}$$

	\kactlimport{phiFunction.h}
	\kactlimport{Min25.h}

\section{Pisano period}
	$\pi(n)$ is a period of Fibbonacci sequence modulo n.
	\mbox{$\pi(nm)=\pi(n)\pi(m)$ for $n \perp m$, $\pi(p^k)=p^{k-1}\pi(p)$.}
	\[\pi(p) \left\{
	\begin{array}{ll}
		=3 & p=2\\
		=20 & p=5\\
		\mid p-1 & p\equiv_{10} \pm 1 \\
		\mid 2(p+1) & p\equiv_{10} \pm 3
	\end{array}\right.\]
	\mbox{$F_i \equiv_p -F_{i+p+1}$ for $p\equiv_{10} \pm 3$. $\pi(n) \le 4n$ for $n\neq 2\cdot 5^r$.}
\section{Fractions}
	\kactlimport{ContinuedFractions.h}
	\kactlimport{FracBinarySearch.h}

\section{Pythagorean Triples}
 The Pythagorean triples are uniquely generated by
 \[ a=k\cdot (m^{2}-n^{2}),\ \,b=k\cdot (2mn),\ \,c=k\cdot (m^{2}+n^{2}), \]
 with $m > n > 0$, $k > 0$, $m \bot n$, and either $m$ or $n$ even.

\section{Pythagorean Tree}
Primitive Pythagorean triples form infinite ternary tree, where each triple occurs exactly once.
Node is a column vector $(a, b, c = \sqrt{a^2+b^2})$, root is $(3, 4, 5)$, and each child is given by a product of a parent and one of the:
$$
\begin{bmatrix}
	1 & -2 & 2\\
	2 & -1 & 2\\
	2 & -2 & 3
\end{bmatrix},
\begin{bmatrix}
	1 & 2 & 2\\
	2 & 1 & 2\\
	2 & 2 & 3
\end{bmatrix},
\begin{bmatrix}
	-1 & 2 & 2\\
	-2 & 1 & 2\\
	-2 & 2 & 3
\end{bmatrix}
$$


\section{Primes \& primitive roots}
	$(1000002089, \{3, 104, \}), (1000000000000200011, \{6, 105\})$
	There are 78498 primes less than 1\,000\,000.

	%Primitive roots exist modulo any prime power $p^a$, except for $p = 2, a > 2$, and there are $\phi(\phi(p^a))$ many.
	%For $p = 2, a > 2$, the group $\mathbb Z_{2^a}^\times$ is instead isomorphic to $\mathbb Z_2 \times \mathbb Z_{2^{a-2}}$.

\section{Estimates}
	% $\sum_{d|n} d = O(n \log \log n)$.

	The number of divisors of $n$ is at most around 100 for $n < 5e4$, 500 for $n < 1e7$, 2000 for $n < 1e10$, 200\,000 for $n < 1e19$.

\section{Mobius Function}
\[
	\mu(n) = \begin{cases} 0 & n \textrm{ is not square free}\\ 1 & n \textrm{ has even number of prime factors}\\ -1 & n \textrm{ has odd number of prime factors}\\\end{cases}
\]
  Mobius Inversion:
  \[ g(n) = \sum_{d|n} f(d) \Leftrightarrow f(n) = \sum_{d|n} \mu(d)g(n/d) \]
  Other useful formulas/forms:

  $ \sum_{d | n} \mu(d) = [ n = 1] $ (very useful)

 $ g(n) = \sum_{1 \leq m \leq n} f(\left\lfloor\frac{n}{m}\right \rfloor ) \Leftrightarrow f(n) = \sum_{1\leq m\leq n} \mu(m)g(\left\lfloor\frac{n}{m}\right\rfloor)$

  Define Dirichlet convolution as $f * g(n) = \sum_{d|n}f(d)g(n/d)$.
  Let $s_f(n) = \sum_{i=1}^n{f(i)}$.
  Then $s_f(n)g(1) = s_{f*g}(n) - \sum^n_{d=2}s_f(\lfloor \frac{n}{d} \rfloor) g(d)$.
